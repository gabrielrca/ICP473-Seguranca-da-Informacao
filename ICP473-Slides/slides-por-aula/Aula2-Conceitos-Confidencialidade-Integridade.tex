\title{Aula 2 - Conceitos Básicos de Segurança da Informação}

\author{Prof. Gabriel Rodrigues Caldas de Aquino}

\institute
{
  Instituto de Computação \\
  Universidade Federal do Rio de Janeiro \\
  gabrielaquino@ic.ufrj.br% Your institution for the title page
}
\date{Compilado em: \\ \today} % Date, can be changed to a custom date

%----------------------------------------------------------------------------------------
%    PRESENTATION SLIDES
%----------------------------------------------------------------------------------------


\begin{frame}
  % Print the title page as the first slide
  \titlepage
\end{frame}



\begin{frame}{Conceitos de Segurança da Informação}
  \textbf{Computer Security (COMPUSEC) (NISTIR 7298 - \href{https://csrc.nist.gov/pubs/ir/7298/r3/final}{Link}):}
  \begin{block}{}
    Medidas e controles que garantem a \textbf{confidencialidade}, \textbf{integridade} e \textbf{disponibilidade} dos ativos de sistemas de informação, incluindo hardware, software, firmware e as informações processadas, armazenadas e comunicadas.
  \end{block}

  \vspace{0.5cm}


\end{frame}
\begin{frame}{Confidencialidade}



  \begin{itemize}
    \item \textbf{Confidencialidade dos dados:} Garante que informações privadas ou confidenciais não sejam disponibilizadas ou divulgadas a indivíduos não autorizados.
    \item \textbf{Privacidade:} Garante que os indivíduos tenham controle ou influência sobre quais informações relacionadas a eles podem ser coletadas, armazenadas e divulgadas, e por quem.
  \end{itemize}

\end{frame}

\begin{frame}{Confidencialidade}
  \textbf{Confidencialidade} é o sigilo de informações ou recursos.

  \vspace{0.4cm}
  A necessidade de manter informações em segredo surge do uso de computadores em áreas sensíveis - Ex: Governo, indústria e outras áreas.

  \vspace{0.4cm}
  \begin{itemize}
    \item Instituições militares e civis do governo aplicam o princípio do \textbf{"necessário saber"}.
    \item Empresas protegem projetos proprietários contra concorrência desleal.
    \item Todas as organizações mantêm registros pessoais em sigilo.
  \end{itemize}
\end{frame}

\begin{frame}{Confidencialidade e Controle de Acesso}
  Mecanismos de \textbf{controle de acesso} são essenciais para garantir a confidencialidade.

  \vspace{0.3cm}
  Um dos principais mecanismos é a \textbf{criptografia}, que embaralha os dados para torná-los incompreensíveis sem a chave correta.

  \vspace{0.3cm}
  \textbf{Exemplo:}
  \begin{itemize}
    \item Um sistema cifra uma declaração de imposto de renda.
    \item Apenas quem possui a \textbf{chave criptográfica} pode decifrá-la e acessar os dados.
    \item Porém, se a chave for exposta durante seu uso, a confidencialidade é comprometida.
  \end{itemize}

  \vspace{0.2cm}
  \textit{A proteção da chave é tão crítica quanto a proteção da própria informação.}
\end{frame}

\begin{frame}{Confidencialidade Além dos Dados}
  \textbf{Confidencialidade} não se aplica apenas ao conteúdo dos dados, mas também à \textbf{existência da informação}.

  \vspace{0.3cm}
  \textbf{Exemplos:}
  \begin{itemize}
    \item Saber que uma pesquisa foi realizada pode ser mais revelador que seu resultado.
    \item O conhecimento de que houve perseguição por uma agência pode ser mais crítico que os detalhes dessa perseguição.
  \end{itemize}

  \vspace{0.3cm}
  \textbf{Ocultação de Recursos:}
  \begin{itemize}
    \item Organizações podem esconder sua infraestrutura para evitar abusos ou exposição.

  \end{itemize}

  \vspace{0.3cm}
  \textit{Mecanismos de controle de acesso podem ocultar tanto dados quanto sua própria existência.}
\end{frame}

\begin{frame}{VeraCrypt: Criptografia de Disco}
  \textbf{VeraCrypt} é uma ferramenta de código aberto para \textbf{criptografia de dados em disco}, derivada do TrueCrypt.

  \vspace{0.3cm}
  \textbf{Principais Características:}
  \begin{itemize}
    \item Criptografia em tempo real (\textit{on-the-fly encryption});
    \item Suporte a algoritmos robustos: AES, Serpent, Twofish, e combinações;
    \item Possibilidade de criar volumes ocultos para negar a existência dos dados (\textit{plausible deniability});
    \item Funciona em Windows, Linux e macOS.
  \end{itemize}

  \vspace{0.3cm}
  \textbf{Casos de Uso:}
  \begin{itemize}
    \item Proteger HDs externos, pendrives e partições do sistema;
    \item Garantir confidencialidade em notebooks e dispositivos móveis;
    \item Armazenar backups criptografados de forma segura.
  \end{itemize}

  \href{https://g1.globo.com/politica/noticia/2010/06/nem-fbi-consegue-decifrar-arquivos-de-daniel-dantas-diz-jornal.html}{Link: Caso do uso TrueCrypt}
\end{frame}



\begin{frame}{Confidencialidade - Caso Hospitalar}
  \begin{block}{\textbf{Proteção de Dados Médicos com Criptografia}}
    \begin{itemize}
      \item \textbf{Cenário Completo:}
            \begin{itemize}
              \item[$\circ$] Rede de 5 hospitais compartilhando exames de imagem (raio-X, ressonâncias)
              \item[$\circ$] Dados classificados como máxima sensibilidade
            \end{itemize}

      \item \textbf{Solução Multicamadas:}
            \begin{enumerate}
              \item Criptografia AES-256 em repouso (armazenamento)
              \item TLS 1.3 com certificados X.509 em trânsito
              \item Chaves protegidas por HSM (Hardware Security Module)
            \end{enumerate}

      \item \textbf{Exemplo Detalhado:} Fluxo de um laudo de câncer de mama:

            \begin{tabular}{|l|l|}
              \hline
              \textbf{Etapa}       & \textbf{Proteção Aplicada}      \\ \hline
              Digitação médico     & Criptografia de disco BitLocker \\ \hline
              Transmissão de dados & Tunelamento VPN + TLS           \\ \hline
              Armazenamento nuvem  & Criptografia e compliance       \\ \hline
            \end{tabular}


    \end{itemize}
  \end{block}


\end{frame}
\begin{frame}{Confidencialidade - Modelo de Permissões em Sistema Acadêmico}
  \begin{block}{\textbf{Hierarquia de Acesso às Notas}}
    \begin{itemize}
      \item \textbf{Arquitetura:} Controle por papéis (RBAC) com políticas granulares

            \vspace{0.3cm}
            \begin{table}[ht]
              \centering
              \begin{tabular}{|l|c|c|c|}
                \hline

                \textbf{Perfil} & \textbf{Próprias Notas} & \textbf{Notas da Turma} & \textbf{Todas Notas} \\ \hline
                Aluno           & \checkmark              & --                      & --                   \\ \hline
                Professor       & \checkmark              & \checkmark              & --                   \\ \hline
                Coordenação     & \checkmark              & \checkmark              & \checkmark           \\ \hline
              \end{tabular}
            \end{table}

      \item \textbf{Exemplo Prático:}
            \begin{itemize}
              \item[•] Aluno João visualiza apenas suas notas em Álgebra
              \item[•] Professora Maria vê todas as notas da turma de Cálculo II
              \item[•] Coordenador Pedro acessa boletins de todos os alunos do curso
            \end{itemize}
    \end{itemize}
  \end{block}


\end{frame}


\begin{frame}{Integridade}

  \begin{itemize}
    \item \textbf{Integridade dos dados:} Garante que informações e programas sejam modificados apenas de maneira autorizada e específica.
    \item \textbf{Integridade do sistema:} Garante que o sistema desempenhe sua função prevista de forma íntegra, livre de manipulações não autorizadas, intencionais ou acidentais.
  \end{itemize}

\end{frame}
\begin{frame}{Integridade: Confiança nos Dados}
  \textbf{Integridade} refere-se à \textbf{confiabilidade dos dados ou recursos}, geralmente no sentido de prevenir modificações indevidas ou não autorizadas.

  \vspace{0.3cm}
  \textbf{Dimensões da Integridade:}
  \begin{itemize}
    \item \textbf{Integridade dos dados:} garante que o conteúdo da informação não foi alterado indevidamente.
    \item \textbf{Integridade da origem (autenticidade):} assegura que a fonte dos dados é confiável e legítima.
  \end{itemize}

  \vspace{0.3cm}
  A integridade da origem afeta a \textbf{credibilidade}, que é fundamental para o funcionamento adequado de um sistema.

  \vspace{0.3cm}
  \textbf{Exemplo:} \\
  Um jornal publica uma informação obtida por vazamento da Casa Branca, mas atribui a fonte incorretamente. \\
  $\Rightarrow$ \textbf{Integridade dos dados} preservada; \textbf{integridade da origem} comprometida.
\end{frame}

\begin{frame}{Mecanismos de Integridade}
  Os mecanismos de integridade dividem-se em duas classes: \textbf{mecanismos de prevenção} e \textbf{mecanismos de detecção}.

  \vspace{0.3cm}
  \textbf{Mecanismos de prevenção} mantêm a integridade dos dados bloqueando tentativas não autorizadas de alterar os dados ou modificá-los de formas não autorizadas.

  \vspace{0.3cm}
  Distinção importante:
  \begin{itemize}
    \item Tentativa de alterar dados sem autorização (ex: invasor tentando modificar dados).
    \item Tentativa de alterar dados de formas não autorizadas por usuários autorizados (ex: contador desviando dinheiro para conta no exterior).
  \end{itemize}

  \vspace{0.3cm}
  \textbf{Mecanismos de detecção} indicam que a integridade dos dados foi comprometida.

  \begin{itemize}
    \item Analisar eventos do sistema (ações de usuários ou do sistema) para detectar problemas.
    \item Analisar os próprios dados para verificar se as restrições esperadas ainda são válidas.
  \end{itemize}

  Podem indicar a causa específica da violação ou apenas reportar que o arquivo está corrompido.


\end{frame}

\begin{frame}{Integridade vs. Confidencialidade}
  Trabalhar com integridade é muito diferente de trabalhar com confidencialidade.

  \vspace{0.3cm}
  Na confidencialidade, os dados ou foram comprometidos ou não.

  \vspace{0.3cm}
  Já a integridade inclui tanto a \textbf{correção} quanto a \textbf{confiabilidade} dos dados.

  \vspace{0.3cm}
  Fatores que afetam a integridade dos dados:
  \begin{itemize}
    \item A origem dos dados (como e de quem foram obtidos).
    \item O quão bem os dados foram protegidos antes de chegar à máquina atual.
    \item O quão bem os dados são protegidos na máquina atual.
  \end{itemize}

  \vspace{0.3cm}
  Avaliar a integridade é difícil, pois depende de suposições sobre a origem dos dados e a confiança nessa origem — aspectos fundamentais de segurança que muitas vezes são negligenciados.
\end{frame}
\begin{frame}[fragile]{Verificando a Integridade de Arquivos Baixados}
  Ao baixar arquivos da internet, é comum verificar sua integridade para garantir que:

  \begin{itemize}
    \item O arquivo não foi corrompido durante a transferência.
    \item O arquivo não foi adulterado por um atacante.
  \end{itemize}

  \vspace{0.3cm}
  \textbf{Como fazer isso?} Usando funções de hash criptográficas como:

  \begin{itemize}
    \item \texttt{MD5} -- não recomendado hoje em dia mas muito usado.
    \item \texttt{SHA-256} -- mais  recomendado atualmente.
  \end{itemize}

  \vspace{0.3cm}
  \textbf{Exemplo no terminal:}

  \begin{verbatim}
$ sha256sum arquivo.iso
$ md5sum arquivo.iso
\end{verbatim}

  Compare o valor obtido com o publicado pelo site oficial.
\end{frame}


\begin{frame}{Sistema de acadêmico com garantia de integridade: Como garantir}

  \begin{itemize}
    \item \textbf{Registro Oficial}
          \begin{itemize}
            \item[•] Cada professor registra notas com login único e seguro
            \item[•] Toda alteração precisa ser aprovada pela coordenação
            \item[•] Data e hora são registradas automaticamente
          \end{itemize}

    \item \textbf{Proteção contra Alterações}
          \begin{itemize}
            \item[•] Notas lançadas não podem ser apagadas
            \item[•] Mudanças criam um novo registro (como versão de documento)
            \item[•] Relatórios mensais são conferidos pela diretoria
          \end{itemize}
  \end{itemize}


  \begin{exampleblock}{Exemplo Prático: Professor dá nota para um aluno}
    \begin{enumerate}
      \item O sistema registra:
            \begin{itemize}
              \item Quem lançou: Professora Ana
              \item Quando: 15/03/2023 às 14:30
              \item Versão do registro: 001
            \end{itemize}
      \item Se precisar corrigir, cria-se um novo registro (versão 002)
    \end{enumerate}
  \end{exampleblock}
\end{frame}

