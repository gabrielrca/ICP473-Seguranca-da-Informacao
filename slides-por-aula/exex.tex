
\begin{frame}{Exercício 1: Cifra de Substituição Simples com Frase-Chave}
\textbf{Ideia:}  
Cada letra do texto plano é substituída por outra letra, usando um \textbf{alfabeto substituto} gerado a partir de uma frase.

\medskip
\textbf{Passos:}
\begin{enumerate}
    \item Escolher uma chave (frase).  
    \item Remover letras repetidas, mantendo apenas a primeira ocorrência.  
    \item Completar o alfabeto com as letras que não apareceram na chave.  
\end{enumerate}

\medskip
\textbf{Exemplo:}  
\begin{itemize}
    \item Texto plano: \texttt{abcdefghijklmnopqrstuvwxyz}  
    \item Chave: \texttt{segurança da informação}  
    \item Alfabeto cifrado: \texttt{segurancdifom...}  
\end{itemize}
\end{frame}

\begin{frame}{Exercício 1:  Decriptação com Substituição}
Um modo de solucionar o problema de distribuição de chave é usar uma linha de um livro que o emissor e o receptor possuem. Normalmente, pelo menos em romances de espionagem, a primeira sentença de um livro serve como chave. 

\medskip
O esquema em particular discutido neste problema é de um dos melhores romances de suspense envolvendo códigos secretos, \textit{Talking to Strange Men}, de Ruth Rendell. 

\medskip
\textbf{Exercício:}  
Descubra essa frase:

\begin{block}{Texto cifrado}
\centering
\texttt{SIDKHKDM AF HCRKIABIE SHIMC KD LFEAILA}
\end{block}

\textbf{Dica:} O texto foi cifrado usando a primeira sentença de \textit{The Other Side of Silence} (sobre o espião Kim Philby):

\smallskip
\emph{``the snow lay thick on the steps and the snowflakes driven by the wind looked black in the headlights of the cars''}
\end{frame}


\begin{frame}{Exercício 1:  Resposta}
\textbf{Sentença usada como chave:} \\
\emph{``the snow lay thick on the steps and the snowflakes driven by the wind looked black in the headlights of the cars''}

\medskip
\textbf{Letras únicas na ordem de aparição:} \\
t, h, e, s, n, o, w, l, a, y, i, c, k, p, d, r, v, u, b, g, f

\medskip
\textbf{Tabela de substituição:}


\footnotesize
\begin{tabular}{ccccccccccccccccccccccccccc}
 a & b & c & d & e & f & g & h & i & j & k & l & m & n & o & p & q & r & s & t & u & v & w & x & y & z \\
\hline
t & h & e & s & n & o & w & l & a & y & i & c & k & p & d & r & v & f & b & g & j & m & q & u & x & z \\
\end{tabular}


\medskip
\begin{block}{Texto decifrado}
\centering
\texttt{BASILISK TO LEVIATHAN BLAKE IS CONTACT}
\end{block}
\end{frame}
