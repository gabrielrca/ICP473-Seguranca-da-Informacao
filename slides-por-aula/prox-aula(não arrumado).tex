\begin{frame}{Terminologia Básica em Segurança da Informação}
\begin{itemize}
    \item \textbf{Adversário (Agente de Ameaça)}:
Indivíduo, grupo, organização ou governo que conduz ou pretende conduzir atividades prejudiciais.


\item \textbf{Recurso do Sistema (Ativo)}:
Aplicação, sistema, equipamento, pessoal ou grupo lógico de sistemas essenciais para a missão.
 




\item \textbf{Política de Segurança}:
Um conjunto de critérios para a prestação de serviços de segurança. Ele define e restringe as atividades de uma instalação de processamento de dados para manter uma condição de segurança para sistemas e dados.

 \item \textbf{Contramedida}:
Um dispositivo ou técnica que tem como objetivo prejudicar a efetividade operacional de atividades indesejadas ou adversárias, ou a prevenção de espionagem, sabotagem, roubo, ou acesso ou uso não autorizado de informações sensíveis ou sistemas de informação.

\end{itemize}

\end{frame}




\begin{frame}{Terminologia Básica em Segurança da Informação (continuação)}
\begin{itemize}
  




\item \textbf{Vulnerabilidade}:
Fraqueza em um sistema de informação, procedimentos de segurança do sistema, controles internos ou implementação que possa ser explorada ou acionada por uma fonte de ameaça.



\item \textbf{Ameaça}:
Qualquer circunstância ou evento com potencial para impactar negativamente as operações organizacionais (incluindo missão, funções, imagem ou reputação), ativos organizacionais, indivíduos, outras organizações ou a Nação por meio de um sistema de informação, através de acesso não autorizado, destruição, divulgação, modificação de informações e/ou negação de serviço.

\item \textbf{Ataque}:
Atividade maliciosa que tenta coletar, interromper, negar, degradar ou destruir recursos ou informações do sistema.

  \item \textbf{Risco}:
Medida da extensão em que um evento pode ameaçar uma entidade, considerando impactos adversos e probabilidade.

\end{itemize}

\end{frame}

\begin{frame}{Diferença entre Ameaça e Ataque}

\begin{block}{Ameaça}
Qualquer circunstância ou evento com potencial para causar dano à segurança do sistema, ativos ou informações.  
\textit{Exemplo:} Tentativa de invasão por um hacker (potencial perigo).
\end{block}

\begin{block}{Ataque}
Ação maliciosa concreta que explora vulnerabilidades para causar dano ou violar a segurança.  
\textit{Exemplo:} Hacker invadindo o sistema e roubando dados (ação realizada).
\end{block}

\vspace{0.5cm}
\textbf{Resumo:}  
\begin{itemize}
    \item \textbf{Ameaça} = Potencial perigo.  
    \item \textbf{Ataque} = Ação efetiva que causa dano.
\end{itemize}

\end{frame}


\begin{frame}{Cenário Prático em Segurança da Informação}

\textbf{Contexto:} Uma universidade possui um sistema online de registro de notas dos alunos.

\vspace{0.3cm}

\begin{itemize}
  \item \textbf{Adversário (Agente de Ameaça):}  
  Um ex-aluno insatisfeito que deseja vazar notas confidenciais para prejudicar a reputação da universidade.

  \item \textbf{Recurso do Sistema (Ativo):}  
  O sistema de gestão acadêmica que armazena e processa as notas dos alunos.

  \item \textbf{Política de Segurança:}  
  Regras que determinam que somente alunos, professores e a coordenação podem acessar as notas e que todos os acessos são registrados e auditados.
  \end{itemize}
\end{frame}
\begin{frame}{Cenário Prático em Segurança da Informação (Continuação)}
\begin{itemize}
    \item \textbf{Vulnerabilidade:}  
  Senha fraca usada por um funcionário da secretaria, permitindo acesso não autorizado ao sistema.

  \item \textbf{Ameaça:}  
  O risco de que o aluno consiga acessar e alterar as notas dos alunos sem permissão.

  \item \textbf{Ataque:}  
  O aluno usa credenciais roubadas para entrar no sistema, acessar e alterar dados confidenciais.

  \item \textbf{Contramedida:}  
  Implementação de autenticação multifator (MFA) e monitoramento de acessos suspeitos.

  \item \textbf{Risco:}  
  A combinação da vulnerabilidade (senha fraca) e da ameaça (aluno malicioso) gera um risco alto de vazamento de dados.
\end{itemize}

\end{frame}


\begin{frame}{Modelo de Segurança da Informação}
\begin{block}{Ativos do Sistema}
Recursos ou ativos que usuários e proprietários desejam proteger:
\begin{itemize}
  \item \textbf{Hardware:} Computadores, dispositivos de armazenamento e comunicação.
  \item \textbf{Software:} Sistemas operacionais, utilitários e aplicações.
  \item \textbf{Dados:} Arquivos, bancos de dados e dados sensíveis (ex: senhas).
  \item \textbf{Comunicações:} Redes locais ou remotas, roteadores, pontes, etc.
\end{itemize}
\end{block}
\end{frame}

\begin{frame}{Vulnerabilidades e Propriedades de Segurança}
\begin{block}{Tipos de Vulnerabilidades}
\begin{itemize}
  \item \textbf{Corrupção:} Modificação indevida dos dados (violação da \textbf{integridade}).
  \item \textbf{Vazamento:} Acesso não autorizado a informações (violação da \textbf{confidencialidade}).
  \item \textbf{Indisponibilidade:} Sistema inacessível ou muito lento (violação da \textbf{disponibilidade}).
\end{itemize}
\end{block}

\begin{block}{Conceitos Associados}
\begin{itemize}
  \item \textbf{Ameaça:} Potencial violação de segurança.
  \item \textbf{Ataque:} Ato concreto que explora uma vulnerabilidade.
  \item \textbf{Agente de ameaça:} Responsável por conduzir o ataque.
\end{itemize}
\end{block}
\end{frame}

\begin{frame}{Classificação de Ataques e Contramedidas}
\begin{block}{Tipos de Ataque}
\begin{itemize}
  \item \textbf{Ativo:} Modifica recursos do sistema ou seu funcionamento.
  \item \textbf{Passivo:} Observa dados sem afetar os recursos.
\end{itemize}

\begin{itemize}
  \item \textbf{Interno:} Realizado por usuários autorizados que abusam dos acessos.
  \item \textbf{Externo:} Realizado por usuários não autorizados, fora da rede.
\end{itemize}
\end{block}

\begin{block}{Contramedidas}
Ações tomadas para prevenir, detectar ou recuperar ataques.  
Podem reduzir o risco, mas também introduzir novas vulnerabilidades.  
\textbf{Risco residual} é o risco que permanece após a adoção de contramedidas.
\end{block}
\end{frame}

\begin{frame}{Nível de Impacto: \textit{Baixo}}

\textbf{Definição:}  
A perda pode ter um efeito adverso limitado nas operações da organização, em seus ativos ou em indivíduos.

\vspace{0.5em}
\begin{block}{Exemplos de Efeitos Adversos Limitados}
\begin{itemize}
    \item Degradação na capacidade de missão, com redução perceptível na eficácia das funções.
    \item Danos menores aos ativos da organização.
    \item Perda financeira de pequeno impacto.
    \item Prejuízo leve a indivíduos.
\end{itemize}
\end{block}

\end{frame}


\begin{frame}{Nível de Impacto: \textit{Moderado}}

\textbf{Definição:}  
A perda pode ter um efeito adverso sério nas operações da organização, em seus ativos ou em indivíduos.

\vspace{0.5em}
\begin{block}{Exemplos de Efeitos Adversos Sérios}
\begin{itemize}
    \item Degradação significativa da capacidade de missão, com redução efetiva das funções principais.
    \item Danos significativos aos ativos da organização.
    \item Perda financeira significativa.
    \item Prejuízo relevante a indivíduos, sem envolver morte ou ferimentos com risco de vida.
\end{itemize}
\end{block}

\end{frame}

\begin{frame}{Nível de Impacto: \textit{Alto}}

\textbf{Definição:}  
A perda pode ter um efeito adverso severo ou catastrófico nas operações da organização, em seus ativos ou em indivíduos.

\vspace{0.5em}
\begin{block}{Exemplos de Efeitos Adversos Severos ou Catastróficos}
\begin{itemize}
    \item Perda ou degradação severa da capacidade de missão, impedindo a realização de uma ou mais funções principais.
    \item Danos de grande escala aos ativos da organização.
    \item Perda financeira significativa.
    \item Prejuízo severo a indivíduos, incluindo risco à vida ou ferimentos com risco de morte.
\end{itemize}
\end{block}

\end{frame}

\begin{frame}{Confidencialidade - Classificação de Informações Acadêmicas}


\begin{block}{\textbf{Informações com diferentes níveis de confidencialidade:}}
\begin{itemize}
  \item \textbf{Alta Confidencialidade:} Informações de saúde dos alunos
  \begin{itemize}
    \item Protegidas por regulamento ou lei
    \item Acesso restrito a alunos, pais (ou responsáveis) e funcionários autorizados
  \end{itemize}
  
  \item \textbf{Confidencialidade Moderada:} Dados de matrícula
  \begin{itemize}
    \item Ainda sob proteção legal, mas acessados com mais frequência
    \item Menor impacto em caso de vazamento
  \end{itemize}
  
  \item \textbf{Baixa Confidencialidade ou Pública:} Listas de diretório (alunos, docentes)
  \begin{itemize}
    \item Frequentemente divulgadas no site da instituição
    \item Consideradas de interesse público
  \end{itemize}
\end{itemize}
\end{block}

\end{frame}

\begin{frame}{Integridade - Caso: Informações de Alergia em Hospital}

\textbf{Cenário:}  
Informações sobre alergias de pacientes armazenadas em um banco de dados hospitalar.

\vspace{0.5em}
\begin{block}{\textbf{Requisitos de Integridade}}
\begin{itemize}
  \item As informações devem estar corretas e atualizadas para garantir tratamentos seguros.
  \item Um médico precisa confiar que os dados de alergia são precisos.
\end{itemize}
\end{block}

\begin{exampleblock}{\textbf{Falha de Integridade e Rastreabilidade}}
\begin{itemize}
  \item Funcionário autorizado (ex: enfermeira) altera deliberadamente os dados para causar dano.
  \item O sistema deve permitir:
    \begin{itemize}
      \item Restauração rápida da base confiável de dados.
      \item Rastreio da ação até o responsável.
    \end{itemize}
\end{itemize}
\end{exampleblock}


\textbf{Conclusão:} Integridade crítica — erro pode causar morte e gerar grande responsabilidade legal para o hospital.

\end{frame}

\begin{frame}{Exemplos de Requisitos de Integridade}
  
  \begin{block}{\textbf{Nível Moderado de Integridade}}
    \begin{itemize}
      \item \textbf{Exemplo:} Fórum online para usuários registrados
      \item \textbf{Riscos:}
      \begin{itemize}
        \item Falsificação de postagens por usuários ou hackers
        \item Defacement do site
      \end{itemize}
      \item \textbf{Impacto Tolerável quando:}
      \begin{itemize}
        \item Fórum apenas para entretenimento
        \item Baixa ou nenhuma receita publicitária
        \item Não utilizado para fins importantes (ex: pesquisa)
      \end{itemize}
      \item \textbf{Consequências:}
      \begin{itemize}
        \item Perda de dados limitada
        \item Impacto financeiro pequeno
        \item Perda de tempo do webmaster
      \end{itemize}
    \end{itemize}
  \end{block}
\end{frame}
\begin{frame}{Exemplos de Requisitos de Integridade (2)}
    
  \begin{block}{\textbf{Nível Baixo de Integridade}}
    \begin{itemize}
      \item \textbf{Exemplo:} Pesquisa online anônima
      \item \textbf{Características:}
      \begin{itemize}
        \item Poucas salvaguardas técnicas
        \item Natureza não-científica reconhecida
        \item Incerteza esperada nos resultados
      \end{itemize}
      \item \textbf{Contexto Típico:}
      \begin{itemize}
        \item Sites de notícias
        \item Enquetes informais
      \end{itemize}
    \end{itemize}
  \end{block}
\end{frame}

\begin{frame}{Disponibilidade - Níveis de Criticidade}

\textbf{Conceito:} Quanto mais crítico o serviço ou componente, maior o nível de disponibilidade exigido.

\vspace{0.5em}
\begin{block}{\textbf{Exemplos por Nível de Disponibilidade}}
\begin{itemize}
  \item \textbf{Alta Disponibilidade:} Serviço de autenticação para sistemas críticos
  \begin{itemize}
    \item Interrupção impede acesso a recursos computacionais e tarefas essenciais
    \item Resulta em perdas financeiras e de produtividade
  \end{itemize}

  \item \textbf{Disponibilidade Moderada:} Website público de universidade
  \begin{itemize}
    \item Afeta a imagem institucional e experiência de alunos e doadores
    \item Não compromete operações internas
  \end{itemize}

  \item \textbf{Baixa Disponibilidade:} Aplicativo de consulta telefônica
  \begin{itemize}
    \item Sua ausência temporária é um incômodo leve
    \item Há alternativas como listas impressas ou centrais telefônicas
  \end{itemize}
\end{itemize}
\end{block}

\begin{itemize}
    \item \textbf{Pergunta:} A criticidade pode ser \textbf{sazonal}?
\end{itemize}

\end{frame}




\begin{frame}{Desafios da Segurança da Informação (1/3)}

\textbf{Segurança e a sua complexidade}

\vspace{0.5em}
\begin{enumerate}
  \item \textbf{Não é tão simples quanto parece:} Conceitos como confidencialidade e integridade parecem diretos, mas os mecanismos para alcançá-los são complexos.
  
  \item \textbf{Ataques inesperados:} É preciso considerar que o atacante pode pensar de forma não convencional e explorar fraquezas sutis.
  
  \item \textbf{Mecanismos contraintuitivos:} Algumas soluções parecem exageradas, até que as ameaças reais sejam compreendidas.
  
  \item \textbf{Decidir onde aplicar segurança:}
    \begin{itemize}
      \item \textit{Fisicamente:} Em quais pontos da rede?
      \item \textit{Logicamente:} Em quais camadas da arquitetura (ex: TCP/IP)?
    \end{itemize}
\end{enumerate}

\end{frame}

\begin{frame}{Desafios da Segurança da Informação (2/3)}

\begin{enumerate}
  \setcounter{enumi}{4}
  \item \textbf{Mais do que algoritmos:} Envolve segredos (ex: chaves criptográficas), que exigem controle de criação, distribuição e proteção.

  \item \textbf{Batalha de inteligência:} O atacante precisa encontrar uma falha. O defensor precisa fechar todas.

  \item \textbf{Valor percebido só após falhas:} Muitas vezes o investimento em segurança só é reconhecido após um incidente.

  \item \textbf{Exige monitoramento contínuo:} Difícil de manter em ambientes sobrecarregados e de curto prazo.
\end{enumerate}

\begin{enumerate}
  \setcounter{enumi}{8}
  \item \textbf{Segurança como pós-requisito:} Muitas vezes é incorporada tardiamente, depois que o sistema já foi projetado.

  \item \textbf{Vista como obstáculo:} Usuários e até administradores veem segurança forte como algo que atrapalha a usabilidade ou eficiência do sistema.
\end{enumerate}

\end{frame}

